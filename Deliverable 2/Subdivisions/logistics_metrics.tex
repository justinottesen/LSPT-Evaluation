\subsection*{Team Members}

\textbf{Member:} Justin Ottesen

\smallskip

\textbf{Role:} Team Lead \& Developer

\smallskip

\textbf{Contributions:} \\
- Set up the GitHub repository with checks, actions, and branch protection rules \\
- Set up the \verb|setup_repo.sh| script to allow for easy setup after cloning \\
- Set up the build system for the component, unit tests, evaltool, and pytest \\
- Writeup of the design specifications \& interface in the README files \\
- Writeup of the \textbf{Team Logistics \& Success Metrics} section \\
- Writeup of the \textbf{Implementation Plan} section

\bigskip

\textbf{Member:} Kenji Chevray

\smallskip

\textbf{Role:} Developer

\smallskip

\textbf{Contributions:} \\
- \color{red} \textbf{TO DO - ADD YOUR CONTRIBUTIONS} \color{black}

\bigskip

\textbf{Member:} Nahuel Fernandez

\smallskip

\textbf{Role:} Developer

\smallskip

\textbf{Contributions:} \\
- \color{red} \textbf{TO DO - ADD YOUR CONTRIBUTIONS} \color{black}


\bigskip

\textbf{Member:} Jayak Patel

\smallskip

\textbf{Role:} Developer

\smallskip

\textbf{Contributions:} \\
- \color{red} \textbf{TO DO - ADD YOUR CONTRIBUTIONS} \color{black}


\subsection*{Team Collaboration Information}
All code and deliverables will be prepared and stored in our \href{https://github.com/justinottesen/LSPT-Evaluation}{GitHub Repository}. There are detailed README files describing our component and plan.

\subsection*{Success Metrics}
\subsubsection*{Internal Success Metrics:}
\medskip
\subsubsection*{Scalability:}
The evaluation component will measure scalability by monitoring the time it takes
for a query to run as the number of users using the search engine increases, and 
also doing another monitor comparing the time it takes for a query to run based on 
the overall storage of the entire search engine is scaled up. This can be done by 
simulating the number of users using the search engine and the number of URLs put 
into the crawler. If the SE seems to run 20-50\% slower as the number of users or 
storage doubles, then the scalability success metric would be at a higher level. 
However, if the SE runs 51-99\% slower as the number of users or storage doubles, 
then the scalability success metric would be at a medium level. Running 100\% 
slower or more as user or storage doubles would show that scalability is low. 

\smallskip

\subsubsection*{Testability:}
- The evaluation component can measure testability by monitoring the total test 
coverage and the success rate of the tests. If there is 91 to 100\% test coverage, 
as well as passing all functional requirements and edge cases, then the 
testability of the code is high level and is considered successful. However, if 
the code does not pass at least 3 of the requirements, or if there is a lower test 
coverage percent, such as 70-90\%, then the testability success metric would be 
considered at a medium level. If the code does not pass more than 3, or if there 
is a lower than 70\% test coverage, then the testability success metric would be 
at a low level.

\smallskip

\subsubsection*{Usability:}
The evaluation component can measure the overall usability of the SE based on the 
usage of the feedback we receive from UI/UX. If UI/UX sends negative feedback(such 
as: “This Search Engine is too slow!”) or bug reports at a rate of less than 1\% 
of the queries searched, this would suggest that there is a relatively high 
usability score. If negative feedback or bug reports are between 1-5\% of the 
queries searched, then it would be considered at a medium level With negative 
feedback suggestions or bug reports that are greater than 5\% of the queries, the 
usability success metric would be low.

\bigskip

\subsection*{External Success Metric:}

\medskip

\subsubsection*{Predictability:}

The evaluation component can measure predictability by measuring how often the 
autofill function is called before a query ID is generated. If there are only 1-2 
calls for the API getAutofill before the API call for getQueryID, then this would 
suggest that a good autofill result was presented. However, if there are 3-5 API 
calls for getAutofill before the API call for getQueryID, then the user likely was 
not satisfied with the autofill results and had to keep typing before finally 
getting a result, suggesting a medium predictability success metric. If there are 
more than 5 API calls for getAutofill before the API call for getQueryID, then the 
predictability success metric would be low.

\smallskip

\subsubsection*{Performance:}
Since the evaluation component sends out the query ID and receives the query ID at 
the end of the query runtime, it can conduct performance tests on the overall 
performance of the query runtime. If the time to run the query is less than 1 
second(what was mentioned in the requirements), then this suggests that the 
performance metric is successful or the performance metric is at a high level. The 
query running between 1 second and 2 seconds would suggest a medium-performance 
metric. The query running for a runtime longer than 2 seconds would be considered 
to have a low-performance metric. 

\smallskip

\subsubsection*{Relevancy:}
Regarding the relevancy success metric, this refers to the link on which the user 
clicks. If the user clicks on the first few links, the search results are 
considered relevant and would have a high score. However, if the user clicks on a 
later link, or even goes to the next page, then the search results would be 
considered irrelevant, and therefore, would not be considered as successful. We 
can measure this based on the average position of the clicked link. If the average 
position of the clicked link is between the first three clicked links, then it 
would be considered a high relevancy score. If the average position of the link 
clicked is between the 4th and 6th positions, then it would be considered a medium 
relevancy score. Finally, if the average relevancy score was lower than the 6th 
position, then the relevant successful metric would be at a low level.

\end{document}