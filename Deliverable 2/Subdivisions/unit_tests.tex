Beware, we are using Test Driven Development. These unit tests will drive functionality, however they are not final, and behavior may change.

\subsection*{JSON Parser}

The first set of tests is for a JSON Parser. Since all of our received messages will be in JSON format, this is incredibly important for our component. A set of unit tests are below (written in the GoogleTest framework which we are using for our component)

\begin{verbatim}
using namespace std;

class JSONParserTest : public ::testing::Test {
protected:
    JSONParser parser;

    // Helper function to test parsing a JSON string and returning the parsed value
    JSONParser::json_value parseJson(const string& json_str) {
        return parser.parse(json_str);
    }
};

// Test parsing a simple string
TEST_F(JSONParserTest, ParseString) {
    string json_str = R"("hello world")";
    auto parsed = parseJson(json_str);
    ASSERT_EQ(std::get<string>(parsed), "hello world");
}

// Test parsing a number
TEST_F(JSONParserTest, ParseNumber) {
    string json_str = "42";
    auto parsed = parseJson(json_str);
    ASSERT_EQ(std::get<double>(parsed), 42);
}

TEST_F(JSONParserTest, ParseNegativeNumber) {
    string json_str = "-42";
    auto parsed = parseJson(json_str);
    ASSERT_EQ(std::get<double>(parsed), -42);
}

TEST_F(JSONParserTest, ParseFloatNumber) {
    string json_str = "3.14";
    auto parsed = parseJson(json_str);
    ASSERT_EQ(std::get<double>(parsed), 3.14);
}

// Test parsing booleans
TEST_F(JSONParserTest, ParseTrue) {
    string json_str = "true";
    auto parsed = parseJson(json_str);
    ASSERT_EQ(std::get<bool>(parsed), true);
}

TEST_F(JSONParserTest, ParseFalse) {
    string json_str = "false";
    auto parsed = parseJson(json_str);
    ASSERT_EQ(std::get<bool>(parsed), false);
}

// Test parsing null
TEST_F(JSONParserTest, ParseNull) {
    string json_str = "null";
    auto parsed = parseJson(json_str);
    ASSERT_EQ(std::get<nullptr_t>(parsed), nullptr);
}

// Test parsing an empty object
TEST_F(JSONParserTest, ParseEmptyObject) {
    string json_str = "{}";
    auto parsed = parseJson(json_str);
    ASSERT_TRUE(std::holds_alternative<map<string, JSONParser::json_value>>(parsed));
    auto& obj = std::get<map<string, JSONParser::json_value>>(parsed);
    ASSERT_TRUE(obj.empty());
}

// Test parsing an object with one key-value pair
TEST_F(JSONParserTest, ParseObjectWithOnePair) {
    string json_str = R"({"key": "value"})";
    auto parsed = parseJson(json_str);
    auto& obj = std::get<map<string, JSONParser::json_value>>(parsed);
    ASSERT_EQ(obj.size(), 1);
    ASSERT_EQ(std::get<string>(obj["key"]), "value");
}

// Test parsing an array with multiple elements
TEST_F(JSONParserTest, ParseArrayWithMultipleElements) {
    string json_str = "[1, 2, 3]";
    auto parsed = parseJson(json_str);
    auto& arr = std::get<vector<JSONParser::json_value>>(parsed);
    ASSERT_EQ(arr.size(), 3);
    ASSERT_EQ(std::get<double>(arr[0]), 1);
    ASSERT_EQ(std::get<double>(arr[1]), 2);
    ASSERT_EQ(std::get<double>(arr[2]), 3);
}

// Test parsing a nested object inside an array
TEST_F(JSONParserTest, ParseArrayWithObject) {
    string json_str = R"([{"key": "value"}, 42])";
    auto parsed = parseJson(json_str);
    auto& arr = std::get<vector<JSONParser::json_value>>(parsed);
    ASSERT_EQ(arr.size(), 2);

    auto& obj = std::get<map<string, JSONParser::json_value>>(arr[0]);
    ASSERT_EQ(obj.size(), 1);
    ASSERT_EQ(std::get<string>(obj["key"]), "value");

    ASSERT_EQ(std::get<double>(arr[1]), 42);
}

// Test parsing a nested array inside an object
TEST_F(JSONParserTest, ParseObjectWithArray) {
    string json_str = R"({"numbers": [1, 2, 3]})";
    auto parsed = parseJson(json_str);
    auto& obj = std::get<map<string, JSONParser::json_value>>(parsed);
    ASSERT_EQ(obj.size(), 1);

    auto& arr = std::get<vector<JSONParser::json_value>>(obj["numbers"]);
    ASSERT_EQ(arr.size(), 3);
    ASSERT_EQ(std::get<double>(arr[0]), 1);
    ASSERT_EQ(std::get<double>(arr[1]), 2);
    ASSERT_EQ(std::get<double>(arr[2]), 3);
}

// Test parsing an object with multiple key-value pairs
TEST_F(JSONParserTest, ParseObjectWithMultiplePairs) {
    string json_str = R"({"name": "Alice", "age": 25, "isStudent": false})";
    auto parsed = parseJson(json_str);
    auto& obj = std::get<map<string, JSONParser::json_value>>(parsed);
    
    ASSERT_EQ(obj.size(), 3);
    ASSERT_EQ(std::get<string>(obj["name"]), "Alice");
    ASSERT_EQ(std::get<double>(obj["age"]), 25);
    ASSERT_EQ(std::get<bool>(obj["isStudent"]), false);
}

// Test parsing invalid JSON
TEST_F(JSONParserTest, ParseInvalidJSON) {
    // Missing closing quote in string
    string json_str = R"("missing closing quote)";
    EXPECT_THROW(parseJson(json_str), std::invalid_argument);

    // Invalid character (unescaped quotes inside string)
    json_str = R"("invalid " quote")";
    EXPECT_THROW(parseJson(json_str), std::invalid_argument);

    // Missing comma in object
    json_str = R"({"name": "Alice" "age": 25})";
    EXPECT_THROW(parseJson(json_str), std::invalid_argument);

    // Invalid number
    json_str = "3.14.15";
    EXPECT_THROW(parseJson(json_str), std::invalid_argument);

    // Unexpected end of input
    json_str = R"({"name": "Alice")";
    EXPECT_THROW(parseJson(json_str), std::invalid_argument);
}
\end{verbatim}

\subsection*{SQLite Wrapper}

We will be abstracting the SQLite functionality into a wrapper class, as this is the main datastore for our component. This testing is integral to ensuring we process data as planned, and have no surprises when interacting with other components.

\begin{verbatim}
// Test opening and closing of the database
TEST_F(SQLiteWrapperTest, OpenDatabase) {
    // In-memory database should open without errors
    EXPECT_NO_THROW(SQLiteWrapper(":memory:"));
}

// Test basic SQL execution (INSERT)
TEST_F(SQLiteWrapperTest, ExecuteQuery) {
    std::string createTableSQL = "CREATE TABLE test(id INTEGER PRIMARY KEY, name TEXT);";
    EXPECT_NO_THROW(db->execute(createTableSQL));

    std::string insertSQL = "INSERT INTO test(name) VALUES ('Alice');";
    EXPECT_NO_THROW(db->execute(insertSQL));

    std::string selectSQL = "SELECT name FROM test WHERE id = 1;";
    auto result = db->query(selectSQL);
    ASSERT_EQ(result.size(), 1);
    ASSERT_EQ(result[0].at("name"), "Alice");
}

// Test query with multiple rows (SELECT)
TEST_F(SQLiteWrapperTest, SelectMultipleRows) {
    std::string createTableSQL = "CREATE TABLE test(id INTEGER PRIMARY KEY, name TEXT);";
    EXPECT_NO_THROW(db->execute(createTableSQL));

    std::string insertSQL1 = "INSERT INTO test(name) VALUES ('Alice');";
    std::string insertSQL2 = "INSERT INTO test(name) VALUES ('Bob');";
    EXPECT_NO_THROW(db->execute(insertSQL1));
    EXPECT_NO_THROW(db->execute(insertSQL2));

    std::string selectSQL = "SELECT name FROM test;";
    auto result = db->query(selectSQL);
    ASSERT_EQ(result.size(), 2);
    ASSERT_EQ(result[0].at("name"), "Alice");
    ASSERT_EQ(result[1].at("name"), "Bob");
}

// Test invalid SQL execution (e.g., syntax error)
TEST_F(SQLiteWrapperTest, InvalidSQL) {
    std::string invalidSQL = "INVALID SQL";
    EXPECT_THROW(db->execute(invalidSQL), std::runtime_error);
}

// Test query for non-existing table (SELECT)
TEST_F(SQLiteWrapperTest, QueryNonExistingTable) {
    std::string selectSQL = "SELECT * FROM non_existing_table;";
    EXPECT_THROW(db->query(selectSQL), std::runtime_error);
}

// Test empty result set
TEST_F(SQLiteWrapperTest, EmptyQueryResult) {
    std::string createTableSQL = "CREATE TABLE test(id INTEGER PRIMARY KEY, name TEXT);";
    EXPECT_NO_THROW(db->execute(createTableSQL));

    std::string selectSQL = "SELECT name FROM test WHERE id = 999;";
    auto result = db->query(selectSQL);
    ASSERT_EQ(result.size(), 0);
}

// Test database closing automatically with RAII
TEST_F(SQLiteWrapperTest, DatabaseClosedOnDestruction) {
    {
        SQLiteWrapper dbWrapper(":memory:");
        std::string createTableSQL = "CREATE TABLE test(id INTEGER PRIMARY KEY, name TEXT);";
        EXPECT_NO_THROW(dbWrapper.execute(createTableSQL));
    }
    // dbWrapper is out of scope here and the database is automatically closed
    // No exception should be thrown
    EXPECT_NO_THROW(SQLiteWrapper(":memory:"));
}
\end{verbatim}

\subsection*{Monotonic Number Generator}

Another crucial piece of functionality we rely on is a monotonically increasing number generator. This is used to generate Query IDs. The component will not behave properly if we repeat numbers, even in case of program crashes and restarts.

\begin{verbatim}
// Test that the generator starts at 1 and increments correctly
TEST_F(MonotonicNumberGeneratorTest, StartAndIncrement) {
    EXPECT_EQ(generator->getNextNumber(), 1);
    EXPECT_EQ(generator->getNextNumber(), 2);
    EXPECT_EQ(generator->getNextNumber(), 3);
}

// Test that the generator state is saved and restored correctly
TEST_F(MonotonicNumberGeneratorTest, StatePersistence) {
    // Generate a few numbers and save the state
    generator->getNextNumber();  // Should return 1
    generator->getNextNumber();  // Should return 2
    generator->getNextNumber();  // Should return 3

    // Destroy and create a new instance to simulate a restart
    delete generator;
    generator = new MonotonicNumberGenerator(file_path);

    // Ensure state is restored, so the next number should be 4
    EXPECT_EQ(generator->getNextNumber(), 4);
}

// Test that the generator works correctly when the file doesn't exist initially
TEST_F(MonotonicNumberGeneratorTest, NoFileInitially) {
    // Ensure the generator starts at 1
    EXPECT_EQ(generator->getNextNumber(), 1);
}

// Test that the generator doesn't return the same number twice
TEST_F(MonotonicNumberGeneratorTest, NoRepeatedNumbers) {
    long long first = generator->getNextNumber();
    long long second = generator->getNextNumber();
    EXPECT_NE(first, second);
}

// Test that the generator works across multiple instances (simulate a crash recovery)
TEST_F(MonotonicNumberGeneratorTest, CrashRecovery) {
    // Generate and save state
    generator->getNextNumber();
    generator->getNextNumber();
    generator->getNextNumber();

    delete generator;

    // Simulate recovery by creating a new instance
    generator = new MonotonicNumberGenerator(file_path);

    // The next number after recovery should continue from the previous number
    EXPECT_EQ(generator->getNextNumber(), 4);
}

// Test that the generator can handle concurrency (multiple threads accessing the generator)
TEST_F(MonotonicNumberGeneratorTest, ThreadSafety) {
    const int num_threads = 10;
    std::vector<std::thread> threads;
    std::set<long long> results;

    // Launch multiple threads that get the next number
    for (int i = 0; i < num_threads; ++i) {
        threads.emplace_back([this, &results]() {
            results.insert(generator->getNextNumber());
        });
    }

    // Wait for all threads to finish
    for (auto& t : threads) {
        t.join();
    }

    // Ensure that all the numbers are unique (monotonically increasing)
    EXPECT_EQ(results.size(), num_threads);
}

// Test that the generator throws an exception when the file cannot be opened for writing
TEST_F(MonotonicNumberGeneratorTest, FileWriteError) {
    // Simulate a situation where the file cannot be opened by setting an invalid path
    generator->~MonotonicNumberGenerator();
    generator = new MonotonicNumberGenerator("/invalid/path/test_state.txt");

    EXPECT_THROW(generator->saveState(), std::runtime_error);
} \end{verbatim}